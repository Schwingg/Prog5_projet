\documentclass[12pt, a4paper]{article}

\usepackage[utf8]{inputenc}
\usepackage[francais]{babel}
\usepackage{meeting}
\usepackage{xcolor}
\setcounter{section}{-1}
\title{Journal du projet}


\author{Schwing, Roug\'e, Moulin, Chaput, Toulisse}

\date{\today}

\begin{document}

\maketitle

\section{"To do list"}
\textbf{A faire:}
\begin{enumerate}
\item Mise en place du gitHub \textbf{OK le 04/01/16} 
\item Définir l'en-t\^ete du fichier ELF \textbf{OK le 04/01/16}
\item Affichage de l'en-t\^ete \textbf{OK le 05/01/16} 
\item Affichage de la table des sections et des détails 
	relatifs \'a chaque section
\item Affichage du contenu d'une section
\item Affichage de la table des symboles et des détails relatifs 
	\'a chaque symbole
\item Affichage des tables de réimplantation et des détails relatifs 
	\'a chaque entrée 
\item Renumérotation des sections
\item Correction des symboles
\item Réimplantations de type \textbf{R\_ARM\_ABS*}
\item Réimplantations de type \textbf{R\_ARM\_JUMP24} et 
	\textbf{R\_ARM\_CALL}
\item Interfaçage avec le simulateur \textbf{ARM}
\item Exécution \'a l'aide du simulateur \textbf{arm-eabi-run}
\end{enumerate}

\textbf{Questions aux professeurs:}
\begin{enumerate}
\item 
	\begin{itemize}
		\item Comment faire pour compiler a l'aide de l'outil fourni
		\item Utiliser le dossier fourni lors des cours d'ALM
	\end{itemize}
\item 
	\begin{itemize}
		\item la taille des sections est elle dynamique?
		\item Oui.
	\end{itemize}
\end{enumerate}

\section{Lundi 4 Janvier}

Il s'agit de notre premier jour pour ce projet. Nous avons pris 
connaissance du sujet et des différents aspects qui devront être 
abordés. Nous avons aussi mis en place un projet sur gitHub qui nous 
permettra de synchroniser nos travaux. Nous avons également 
commencé a mettre en place la lecture de l'en-t\^ete des fichier 
ELF.

\section{Mardi 5 Janvier}

Nous commencons cette journée par les questions aux professeurs a propos de 
l`utilisation de l`emulateur ARM qui nous est donn\'e. Nous avons egalement 
mis au propre le code que nous avions fait hier.

Ensuite, nous avons ajouté une structure pour faciliter l'accès aux informations lues dans le header du fichier ELF.
En parallèle, l'étude théorique de l'affichage du tableau des sections a été réalisée. L'iplémentation sera réalisée prochainement.

Enfin, le code existant a été nettoyé et commenté.

\end{document}


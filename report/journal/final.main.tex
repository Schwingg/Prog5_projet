\documentclass[12pt, a4paper]{article}

\usepackage[utf8]{inputenc}
\usepackage[francais]{babel}
\usepackage{meeting}
\usepackage{xcolor}
\setcounter{section}{-1}
\title{Journal du projet}


\author{Schwing, Roug\'e, Moulin, Chaput, Toulisse}

\date{\today}

\begin{document}

\maketitle

\section{"To do list"}
\textbf{A faire:}
\begin{enumerate}
\item Mise en place du gitHub \textbf{OK le 04/01/16} 
\item Définir l'en-t\^ete du fichier ELF \textbf{OK le 04/01/16}
\item Affichage de l'en-t\^ete \textbf{OK le 05/01/16} 
\item Affichage de la table des sections et des détails
	relatifs \'a chaque section \textbf{OK le 06/01/16} 
\item Affichage du contenu d'une section \textbf{OK le 07/01/16} 
\item Affichage de la table des symboles et des détails relatifs 
	\'a chaque symbole \textbf{OK le 07/01/16} 
\item Affichage des tables de réimplantation et des détails relatifs 
	\'a chaque entrée \textbf{OK le 08/01/16} 
\item Renumérotation des sections \textbf{OK le 11/01/16} 
\item Correction des symboles
\item Réimplantations de type \textbf{R\_ARM\_ABS*}
\item Réimplantations de type \textbf{R\_ARM\_JUMP24} et 
	\textbf{R\_ARM\_CALL}
\item Interfaçage avec le simulateur \textbf{ARM}
\item Exécution \'a l'aide du simulateur \textbf{arm-eabi-run}
\end{enumerate}

\textbf{Questions aux professeurs:}
\begin{enumerate}
\item 
	\begin{itemize}
		\item Comment faire pour compiler a l'aide de l'outil fourni
		\item Utiliser le dossier fourni lors des cours d'ALM
	\end{itemize}
\item 
	\begin{itemize}
		\item la taille des sections est elle dynamique?
		\item Oui.
	\end{itemize}
\end{enumerate}

\section{Lundi 4 Janvier}

Il s'agit de notre premier jour pour ce projet. Nous avons pris 
connaissance du sujet et des différents aspects qui devront être 
abordés. Nous avons aussi mis en place un projet sur gitHub qui nous 
permettra de synchroniser nos travaux. Nous avons également 
commencé a mettre en place la lecture de l'en-t\^ete des fichier 
ELF.

\section{Mardi 5 Janvier}

Nous commençons cette journée par les questions aux professeurs a propos de 
l`utilisation de l`emulateur ARM qui nous est donn\'e. Nous avons également 
mis au propre le code que nous avions fait hier.

Ensuite, nous avons ajouté une structure pour faciliter l'accès aux 
informations lues dans l’entête  du fichier ELF.En parallèle, l'étude 
théorique de l'affichage du tableau des sections a été réalisée. 
L'implémentation sera réalisée prochainement.

Enfin, le code existant a été nettoyé et commenté.

\section{Mercredi 6 Janvier}

L'implémentation et le débogage lié au projet de l'affichage du tableau des 
sections a été réalisé.Une première version de l'affichage d'une section 
également, pour l'instant la fonction affiche toutes les sections, sans 
prendre en compte le choix de l'utilisateur (ce qui sera implémenté 
prochainement).

Nous avons également réalisé quelques tests sur nos anciennes fonctions et 
commencé à factoriser le code.

\section{Jeudi 7 Janvier}

Nous avons finit de faire la lecture du contenu des sections. Nous avons 
également commenc\'e a envisager de faire une réunion des ".h". 

Les test des fonctions utiles a la lecture du contenu des sections ont 
également étaient fait dans le but de corriger les erreurs d'affichage ainsi 
que les erreurs de lecture (big/little endian).

Nous avons également fait la mis en place l’affichage des tables de des 
symboles.

\section{Vendredi 8 Janvier}

Nous avons fait la réunifications de tous les ".h" dans un but de 
simplification.

Nous avons mis en place l’affichage des tables de r\'eimplémentation, ce qui 
est utile pour la renumérotation des sections.

Nous avons également commencé le codage et  la compréhension de la 
renumérotation des sections. 

\section{Lundi 11 Janvier}

Mise en place des options pour pouvoir contrôler notre programme ce qui avait 
étais omis precedement.

Fin du codage de la renumérotation des sections et mise en test de cette partie.

Réunion de mi-projet avec le cor enseignent.

\end{document}


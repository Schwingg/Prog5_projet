\documentclass[utf8]{beamer}

\usepackage{tikz}

%%%%%%%%%%%%%%%%%%%%%%%%%%%%%%%%%%%%%%%%%%%%%%%%%%%%%%%%%%%%%%%%%%%%%%%%%%%%%%
%% lstlisting
% trick to use lstlisting and beamer overlay
%%%%%%%%%%%%%%%%%%%%%%%%%%%%%%%%%%%%%%%%%%%%%%%%%%%%%%%%%%%%%%%%%%%%%%%%%%%%%%
\newsavebox{\lstbox}
\usepackage{listings}

%%%%%%%%%%%%%%%%%%%%%%%%%%%%%%%%%%%%%%%%%%%%%%%%%%%%%%%%%%%%%%%%%%%%%%%%%%%%%%
%% hyperfer
%%%%%%%%%%%%%%%%%%%%%%%%%%%%%%%%%%%%%%%%%%%%%%%%%%%%%%%%%%%%%%%%%%%%%%%%%%%%%%
\usepackage{hyperref}

\hypersetup{
  pdfauthor={Nicolas Gibelin},
  pdftitle={Introduction \`a \LaTeX},
  pdfkeywords={\TeX} {\LaTeX} {Initiation},
  colorlinks=true,
  urlcolor=gray,  % external links
  linkcolor=white, % internal links
}

%%%%%%%%%%%%%%%%%%%%%%%%%%%%%%%%%%%%%%%%%%%%%%%%%%%%%%%%%%%%%%%%%%%%%%%%%%%%%%
%% booktabs
%%%%%%%%%%%%%%%%%%%%%%%%%%%%%%%%%%%%%%%%%%%%%%%%%%%%%%%%%%%%%%%%%%%%%%%%%%%%%%
%% The package enhances the quality of tables
\usepackage{booktabs}

%%%%%%%%%%%%%%%%%%%%%%%%%%%%%%%%%%%%%%%%%%%%%%%%%%%%%%%%%%%%%%%%%%%%%%%%%%%%%%
%% Beamer theme and color definition
%%
%% See http://mcclinews.free.fr/latex/introbeamer
%% for a top presentation of all beamer
%%%%%%%%%%%%%%%%%%%%%%%%%%%%%%%%%%%%%%%%%%%%%%%%%%%%%%%%%%%%%%%%%%%%%%%%%%%%%%

%%% Theme
\usetheme[width=2em]{Rochester}

%%% \usetheme{nom du theme global}
%%% \usecolortheme{nom du theme de couleur}
%%% \usefonttheme{nom du theme de police}
%%% \useinnertheme{nom du theme interne}
%%% \useoutertheme{nom du theme externe}

\useoutertheme[width=0pt]{sidebar}
%\setbeamersize{sidebar width left=0pt}

%%% Colors : http://mcclinews.free.fr/latex/introbeamer/les_couleurs.html
\definecolor{themecolor}{HTML}{511123}
\definecolor{themealter}{HTML}{115123}

\usecolortheme[named=themecolor]{structure}

%% Set background of all slides width shading
\setbeamertemplate{background canvas}[vertical shading]%
[top=structure.fg!70,bottom=structure.fg!05]

%% Themes for beamerbox
\beamerboxesdeclarecolorscheme{clair}{themecolor}{themecolor!20}

%% Author(s), titles ....
%% If you want $Rev: 48 $ svn property being replaced in this file, don't forget to
%% had Rev Property to this file : svn propset svn:keywords "Rev" filename
\date{Version 0.1 ($Rev: 48 $)}
\author{Nicolas Gibelin\\%
\href{mailto:Nicolas.Gibelin@imag.fr}{Nicolas.Gibelin@imag.fr}}
\title{Title}
\subtitle{Subtitle}


%%% % Sommaire local. En deux colonnes
%%% \begin{frame}{Plan}
%%%   \begin{columns}[t]
%%%   \begin{column}{5cm}
%%%   \tableofcontents[sections={1-4},currentsection, hideothersubsections]
%%%   \end{column}
%%%   \begin{column}{5cm}
%%%   \tableofcontents[sections={5-8},currentsection,hideothersubsections]
%%%   \end{column}
%%%   \end{columns}
%%% \end{frame}

%% Logo
\logo{\includegraphics[height=0.5cm]{mi2s}}

%% 
%% Insertion of tac at each section/subsection call
%%
%\AtBeginSubsection[]
\AtBeginSection[]
 {
  \begin{frame}<beamer>
    \frametitle{Plan}
%%    \tableofcontents[currentsection,currentsubsection]
    {\tiny \tableofcontents[currentsection,hideothersubsections]}
  \end{frame}
 }

%%%%%%%%%%%%%%%%%%%%%%%%%%%%%%%%%%%%%%%%%%%%%%%%%%%%%%%%%%%%%%%%%%%%%%%%%%%%%%
%% Sommaire en deux colonnes
%%%%%%%%%%%%%%%%%%%%%%%%%%%%%%%%%%%%%%%%%%%%%%%%%%%%%%%%%%%%%%%%%%%%%%%%%%%%%%
%%\AtBeginSection[]
%%{
%%  \begin{frame}{Plan}
%%    \begin{columns}[t]
%%      \begin{column}{5cm}
%%        \tableofcontents[sections={1-4},currentsection, hideothersubsections]
%%      \end{column}
%%      \begin{column}{5cm}
%%        \tableofcontents[sections={5-8},currentsection,hideothersubsections]
%%      \end{column}
%%    \end{columns}
%%  \end{frame} 
%%}

%%%%%%%%%%%%%%%%%%%%%%%%%%%%%%%%%%%%%%%%%%%%%%%%%%%%%%%%%%%%%%%%%%%%%%%%%%%%%%
%% Declare and image for title background image
%%%%%%%%%%%%%%%%%%%%%%%%%%%%%%%%%%%%%%%%%%%%%%%%%%%%%%%%%%%%%%%%%%%%%%%%%%%%%%
\pgfdeclareimage[height=\paperheight,width=\paperwidth]{background}{background.jpg}

\begin{document}

%% Title page with transparent background image (use tikz)
\setbeamertemplate{background canvas}{\tikz\node[opacity=0.3,inner sep=0]{\pgfuseimage{background}};}
\frame{\titlepage}

%% Restore shading
\setbeamertemplate{background canvas}[vertical shading]%
[top=structure.fg!70,bottom=structure.fg!05]
\section{Introduction}

\begin{frame}
\begin{itemize}
  \pause \item Test1
  \pause \item Test2
  \pause \item Test3
  \pause \item Test4
\end{itemize}
\end{frame}

\section{Astuces}

\subsection{lstlisting and beamer overprint}

\begin{lrbox}{\lstbox}
{\tiny\begin{lstlisting}
\begin{itemize}
  \item premier
  \begin{enumerate}
    \item premier sous-item
    \item second sous-item
  \end{enumerate}
  \item second
\end{itemize}
\begin{description}
  \item[Premier] Premier item
  \item[Second] Second item
\end{description}
\end{lstlisting}}
\end{lrbox}

\begin{frame}[fragile]
  \frametitle{lstlisting and beamer overprint}
  \begin{itemize}
      \pause \item itemize
      \pause \item enumerate
      \pause \item description
  \end{itemize}
  \begin{overprint}
    \onslide<5>
    \begin{exampleblock}{\LaTeX}
      {\tiny\begin{itemize}
      \item premier
        \begin{enumerate}
        \item premier sous-item
        \item second sous-item
        \end{enumerate}
      \item second
      \end{itemize}
      \begin{description}
        \item[Premier] Premier item
        \item[Second] Second item
      \end{description}}
    \end{exampleblock}
    \onslide<6>
    \begin{block}{Code \LaTeX}
      {\usebox{\lstbox}}
    \end{block}
    \onslide<7>
    \begin{block}{Red\'efinition}
      Vous pouvez changer les puces et le format de num\'erotation : voir la
      mise en forme avanc\'ee sur \href{http://fr.wikibooks.org/wiki/LaTeX/}{Wikibooks}
    \end{block}
  \end{overprint}
\end{frame}


\subsection{Color theme change}

\setbeamertemplate{background canvas}[vertical shading]%
[top=themealter!05,bottom=themealter!90]
\begin{frame}
\frametitle{Color theme change}
\end{frame}

\subsection{Color theme restore default}

\setbeamertemplate{background canvas}[vertical shading]%
[top=structure.fg!80,bottom=structure.fg!05]
%%\setbeamertemplate{background canvas}[default]
\begin{frame}
\frametitle{Color theme restore default}
\end{frame}

\section{Divers}

\setbeamertemplate{blocks}[rounded]%
[shadow=true]
\begin{frame}
\frametitle{Divers}
\begin{block}{Bloc arrondi et ombré}
  Un bloc avec option rounded et shadow
  Un peu d'ombre en plus\dots
\end{block}

\begin{alertblock}{Un bloc très alerte}
  Texte du block \texttt{alertblock}
\end{alertblock}

\begin{exampleblock}{Un bloc exemplaire}
  Exemple de block \texttt{exampleblock}
\end{exampleblock}

\begin{beamerboxesrounded}%
      [lower=structure, %
       upper=block title,%
       shadow=true]%
      {Beamer Box Rounded}
    Texte à l'intérieur de la boîte arrondie
\end{beamerboxesrounded}

\begin{beamerboxesrounded}%
      [scheme=clair, shadow=true]%
      {Un bloc arrondi}
Texte en boîte arrondie de toutes les couleurs
\end{beamerboxesrounded}

\end{frame}

\begin{frame}
\begin{itemize}
  \item<1-> l'élément de liste apparaîtra depuis la couche numéro 1.
  \item<2-> \textbf<2>{l'élément de liste apparaîtra en gras
            sur la couche 2 puis normalement.}
  \item<3-> l'élément de liste apparaîtra depuis la couche numéro 3.
  \item<2-> l'élément de liste apparaîtra depuis la couche numéro 2.
  \item<4-> l'élément de liste apparaîtra depuis la couche numéro 4.
\end{itemize}
\end{frame}

\begin{frame}
\begin{itemize}[<+->]
    \item l'élément de liste apparaîtra depuis la couche numéro 1.
    \item \textbf<.>{l'élément de liste apparaîtra en gras 
                    sur la couche 2 puis normalement.}
    \item l'élément de liste apparaîtra depuis la couche numéro 3.  
  \end{itemize}
\end{frame}

\section{Questions ?}

\setbeamertemplate{background canvas}[default]
\setbeamercolor{background canvas}{bg=themealter!20}
\begin{frame}
\end{frame}

\end{document}

\documentclass[12pt, a4paper]{article}

\usepackage[utf8]{inputenc}
\usepackage[francais]{babel}
\usepackage{meeting}
\usepackage{xcolor}
\setcounter{section}{-1}
\title{Fonctionnalités du projet}


\author{Schwing, Roug\'e, Moulin, Chaput, Toulisse}

\date{\today}

\begin{document}

\maketitle

\section{Fonctionnalités implémentées}

Les parties 1 à 9 ont été implémentées.
Les options nécessaires à leur activation sont disponibles dans le manuel d'utilisation.


L'option -x (affichage du contenu d'une section) prend obligatoirement un paramètre.
Cette option ne permet donc d'afficher le contenu des sections qu'une par une.


\section{Fonctionnalités non implémentées}

Partie 10: un prototype non fonctionnel est visible dans la branche SimulArm de notre GitHub à l'adresse suivante :

https://github.com/Schwingg/Prog5\_projet


Partie 11: Non réalisée.


\section{Bugs connus}

Les options attendant un paramètre ont un comportement assez restrictif:

Ainsi,lors de l'exécution la commande "./projet -fh ... " l'option -f (attendant un argument) va prendre h (l'option suivante) comme argument et l'éxécution sera dans le meilleur des cas différente de celle attendue.
\end{document}


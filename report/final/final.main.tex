\documentclass[12pt, a4paper]{article}

\usepackage[utf8]{inputenc}
\usepackage[francais]{babel}
\usepackage{meeting}
\usepackage{xcolor}
\setcounter{section}{-1}
\title{Apn\'ee 2: Test de programmes}


\author{Georges Schwing, Christopher Gil}

\date{\today}

\begin{document}

\maketitle

\section{Objectifs}

L'objectif de cette APNEE est de réaliser un script implémentant un 
système de gestion de fichiers mis au rebut similaire à la corbeille 
présente dans la plupart des interfaces graphiques. Les différentes 
parties sont progressives et vous permettent de construire votre 
script de manière incrémentale. Pensez à conserver une copie de 
votre travail à chaque étape afin de réaliser votre compte rendu. 

\section{Première version}

Dans sa première version, le script corbeille devra déplacer tous 
les fichiers dont les noms lui seront donnés en arguments de la 
ligne de commande dans le répertoire suivant:
\begin{lstlisting}
$HOME/.corbeille
\end{lstlisting}
en créant au préalable ce répertoire s'il n'existe pas. Nous 
supposerons que les noms de fichier donnés sont absolus ou relatifs 
au répertoire courant. Attention, n'oubliez pas que les noms de 
fichier peuvent être quelconques et, par exemple, contenir des 
caractères particuliers (espace, ?, *, ...). D'après vous, pourquoi 
le nom du répertoire suivant: 
\begin{lstlisting}
$HOME/.corbeille 
\end{lstlisting} 
comporte-t-il un point ? D'après vous, comment devez vous traiter 
le cas d'arguments qui ne correspondent pas à un nom de fichier ?

\subsection{le script produit}

Le script suivant permet de gérer l'effacement des fichiers passés 
en paramètres. Il permet de gérer les caractères spéciaux et les 
espaces. Ce script vérifie aussi si les dossier ou les fichiers 
passés en paramètres existes ou non et si ils sont accessible en 
écriture (nécessaire pour le déplacement).

\begin{lstlisting}[frame=single, language=bash]
#!/bin/bash
#specifie ici le dossier de la corbeille
trashdir=~/.corbeille


suppr(){
	#rajout de l'antislash necessaire aux tests
	antislash=`echo "$1" | sed "s/\ /\\\ /"`
	if [ -d "$antislash" -o -f "$antislash" ] ; then
		#le fichier ou le dossier existe
		if [ -w "$antislash" ] ; then
			#peut on ecrire le fichioer ( le deplacer)
			mv "$antislash" $trashdir
		else
			echo "Vous n'avez pas les droits en ecriture sur $1"
		fi
	else
		echo "Le fichier ou le dossier $1 n'existe pas"
	fi
}


init(){
	#verification de la non presence du dossier de la corbeille
	#et creation si besoin
	if [ ! -d $trashdir ] ; then
		mkdir $trashdir 
	fi
}

init
#verification de la presence de parametres
if [ $# -eq 0 ] ; then
	#il n'y a pas de parametres
	echo "Vous devez passer des parametres !!!!!"
else
	#il y a des paramaetres
	for ((i=1;i<=$#;i++)) ; do
		suppr "$(eval echo "\$$i")"
	done
fi
\end{lstlisting}

\section{Deuxième version}

Votre script doit maintenant accepter une commande comme premier 
argument. Cette commande pourra être : 
\begin{itemize}
\item \textbf{efface} : dans ce cas, les arguments qui suivent la 
commande correspondront à des noms de fichier à déplacer vers le 
répertoire suivant :
\begin{lstlisting}
$HOME/.corbeille
\end{lstlisting}

\item \textbf{restaure} : dans ce cas, les arguments qui suivent la 
commande correspondront à des noms de fichier à déplacer depuis le 
répertoire suivant :
\begin{lstlisting}
$HOME/.corbeille
\end{lstlisting}
vers le répertoire courant. Le nom de chaque fichier à restaurer 
devra être donné tel qu'il était avant effacement du fichier et ce 
nom sera le nom du fichier une fois restauré (on restaure à 
l'endroit original). Attention si un nom de fichier comporte des 
répertoires intermédiaires, il faudra créer ceux-ci s'ils n'existent 
pas (ou plus)

\item \textbf{info} : dans ce cas, pour chacun des arguments qui 
suivent la commande, il faudra afficher les infos concernant les 
fichiers de nom correspondant (s'ils se trouvent dans la corbeille, 
depuis combien de temps, ...). Si aucun argument n'est donné, il 
faudra afficher des infos sur tous les fichiers contenus dans la 
corbeille

\item \textbf{vide} : dans ce cas, il faudra supprimer de la 
corbeille les fichiers dont le nom sera donné en argument de la 
commande ou vider totalement la corbeille si aucun argument n'est 
donné
\end{itemize}
Attention, pour cette question vous aurez à stocker, sous la forme 
que vous voudrez, des meta informations sur les fichiers de la 
corbeille, en particulier la localisation originale de chaque 
fichier. 

\subsection{Structure du fichier meta}

Nous avons décidés d'utiliser un fichier meta qui contient dans 
l'ordre les informations suivantes :
\begin{itemize}
	\item le dossier contenant le fichier ou le dossier effacé
	\item la date d'effacement du fichier
	\item l’arborescence du fichier ou du dossier d'origine
\end{itemize}
Les informations sont séparés par le caractère espace. Le fait que 
ce caractère est un caractère possible dans une arborescence n'est 
pas un problèmes car l'arborescence est donner en dernier dans la 
ligne du fichier.
\subsection{Le script produit}

Le script produit dans le cadre de la question deux permet toujours 
de gérer les cas spéciaux qui ont était abordé dans la première 
partie.

\end{document}